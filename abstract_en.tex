% English Abstract
% File: abstract_en.tex

\begin{abstract}

This thesis analyzes the enrollment forecasting of higher education institutions in the US by utilizing IPEDS data (2010--2021), which includes 86,798 institution-year observations. The research evaluates naive persistence, moving averages, ARIMA, Ridge regression, and Random Forest methods through walk-forward validation. The main discovery is that enrollment has an extremely high degree of persistence (98\% of the variance is explained by the lagged values), which results in simple baseline models outperforming sophisticated machine learning techniques. The simple naive persistence baseline attains an average MAE of 39.43 students, whereas it is 40.33 (Ridge) and 41.21 (Random Forest). Panel regression shows that admissions and affordability factors are statistically significant; however, their practical impact is limited by the dominance of persistence. The COVID-19 pandemic stress test reveals that simple models are more resilient. The results question the use of machine learning in education analytics and suggest simplicity in operational forecasting, which can be useful for institutional research and strategic planning.

\end{abstract}

\vspace{1em}

\noindent\textbf{Keywords:} Enrollment Forecasting, Higher Education Analytics, Machine Learning, Time Series Analysis, IPEDS Data, Persistence Models, Institutional Research, Predictive Analytics, Ridge Regression, Random Forest, ARIMA Models, Panel Data Analysis, Walk-Forward Validation, Enrollment Management, University Planning, Educational Data Mining, Student Recruitment, COVID-19 Higher Education, Baseline Forecasting, Affordability Indicators, Admissions Funnel, Enrollment Drivers, Financial Aid Impact, Institutional Capacity, Model Comparison, Forecast Accuracy, Strategic Enrollment Planning, Postsecondary Education, Data-Driven Decision Making, Enrollment Persistence
