% Latvian Abstract (Kopsavilkums)
% File: abstract_lv.tex

\begin{abstract}

Šis bakalaura darbs analizē augstākās izglītības iestāžu uzņemšanas prognozēšanu ASV, izmantojot IPEDS datus (2010--2021), kas ietver 86~798 iestāžu-gadu novērojumus. Pētījums novērtē naivo persistenci, slīdošos vidējos, ARIMA, Ridge regresiju un Random Forest metodes, izmantojot soli-pa-solim validāciju. Galvenais atklājums ir tas, ka uzņemšanai ir ārkārtīgi augsta persistences pakāpe (98\% no dispersijas izskaidro aizkavētās vērtības), kas nozīmē, ka vienkāršie bāzes modeļi pārspēj sarežģītās mašīnmācīšanās metodes. Vienkāršais naivās persistences bāzes modelis sasniedz vidējo absolūto kļūdu (MAE) 39,43 studentiem, salīdzinot ar 40,33 (Ridge) un 41,21 (Random Forest). Paneļu regresija parāda, ka uzņemšanas un pieejamības faktori ir statistiski nozīmīgi; tomēr to praktiskā ietekme ir ierobežota dominējošās persistences dēļ. COVID-19 pandēmijas stresa tests atklāj, ka vienkāršie modeļi ir noturīgāki. Rezultāti apšauba mašīnmācīšanās izmantošanu izglītības analītikā un iesaka vienkāršību operatīvajā prognozēšanā, kas var būt noderīga institucionālajiem pētījumiem un stratēģiskajai plānošanai.

\end{abstract}

\vspace{1em}

\noindent\textbf{Atslēgvārdi:} Uzņemšanas prognozēšana, Augstākās izglītības analītika, Mašīnmācīšanās, Laika rindu analīze, IPEDS dati, Persistences modeļi, Institucionālie pētījumi, Prognozējošā analītika, Ridge regresija, Random Forest, ARIMA modeļi, Paneļu datu analīze, Soli-pa-solim validācija, Uzņemšanas pārvaldība, Universitāšu plānošana, Izglītības datu ieguве, Studentu pieņemšana, COVID-19 augstākā izglītība, Bāzes prognozēšana, Pieejamības rādītāji, Uzņemšanas piltuве, Uzņemšanas virzītājspēki, Finansiālās palīdzības ietekme, Institucionālā kapacitāte, Modeļu salīdzinājums, Prognozes precizitāte, Stratēģiskā uzņemšanas plānošana, Pēcvidusskolas izglītība, Datu vadīta lēmumu pieņemšana, Uzņemšanas persistence
